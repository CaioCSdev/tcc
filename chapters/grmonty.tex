\chapter{Grmonty: Monte Carlo para Relatividade Geral}
\label{cap:grmonty}

\section{O que é o GRmonty}
  Dolence et al definem o \textbf{grmonty} como ``software destinado a calcular o espectro de plasmas quentes e opticamente finos levando em consideração a relatividade geral por completo, utilizando um código de transporte radioativo baseado na técnica de Monte Carlo''\citep[p.1, traduzido]{Dolence:09}. Em outras palavras o programa estima o espectro de uma simulação de magnetoidrodinâmica \cite{eletro-hidro-dynamic:42} relativística utilizando o método de Monte Carlo.

  Buracos negros são regiões no espaço-tempo que possuem uma força gravitacional tão grande que nenhuma partícula, nem mesmo a luz, é capaz de escapar dessa área. Uma vez que algo adentra esse local é impossível que saia.

  Quando partículas passam a uma distância razoável de um buraco negro, é possível que sua trajetória seja drasticamente modificada, acarretando em interações que seriam difíceis de ocorrer em condições diferentes. Tais interações podem criar novas partículas, emitir radiação em diversas frequências ou resultar na queda do elemento no buraco negro.

  Quando é dito que o \textbf{grmonty} leva em consideração toda a relatividade geral, significa que o programa busca representar, em sua solução, todos os efeitos que a gravidade pode causar ao tecido do espaço-tempo e como esses efeitos propiciam interações entre as partículas, por fim moldando o espectro produzido

  Utilizando o método de Monte Carlo na geração dos dados, os fótons, e a partir de um dado modelo de plasma fornecido como entrada, o qual especifica velocidade, densidade, força do campo magnético e temperatura, o programa busca gerar o espectro. Para tanto um número próximo a \textit{N} - fornecido na entrada - de fótons é gerado e para cada um sua trajetória é traçada. Nessa trajetória o fóton é espalhado e pode passar por diferentes interações, até finalmente ser mensurado.

  Depois de algumas iterações um número próximo a \textit{N} de fótons já foi gerado e rastreado, assim um relatório com o espectro é obtido e retornado pelo programa que finalmente termina.

\section{Qual o objetivo}
  Foram desenvolvidas várias técnicas para calcular a transferência radioativa a partir de fontes como as descritas abaixo\citep{Dolence:09}, porém poucas levam em conta a relatividade geral como um todo, principalmente no quesito de objetos, fontes, muito massivas ou com velocidades próxima a da luz, o \textbf{grmonty} vem para aprimorar esses cálculos.

  Qualquer fonte astrofísica de radiação que seja relativística, ou seja, qualquer corpo ou fenômeno fonte de radiação eletromagnética, seja do rádio à raios gama e que é relativística: apresenta uma considerável distorção no espaço-tempo, seja por estar em velocidades próximas a da luz, seja por possuir uma enorme quantidade de massa e/ou energia. Exemplos de objetos são os buracos negros e estrelas de nêutrons, fenômenos são os \textit{Gamma Ray Bursts} ou núcleos ativos de galáxias.


\section{Como o programa funciona}
\label{sec:comofaz}

  No momento de criação e rastreio dos fótons o programa faz uso da biblioteca \textbf{OpenMP} para paralelizar o desenvolvimento dos fótons, graças a esta abordagem é viável o potencial uso de todos os núcleos disponíveis na CPU da máquina. A biblioteca é utilizada para que cada fóton seja produzido e espalhado de forma independente dos outros e funcionando em paralelo, além disso todas as instruções não dependem do fóton em si, elas são as mesmas instruções para todos os fótons. Desta forma podemos caracterizar o \textbf{grmonty} como tendo uma computação SIMD.

  ``Única Instrução Múltiplos Dados: Nesse tipo de computação podem haver múltiplos processadores, cada um operando sobre seu item de dados, mas estão todos executando a mesma instrução naquele item de dados''\citep[p.84, traduzido]{HCP:16}. A arquitetura SIMD trabalha em ressonância com o \textbf{OpenMP} uma vez que torna a paralelização muito simples de ser aplicada: não há variáveis compartilhadas, não há condicionais ou desvios de fluxo que tornem cada execução diferente uma da outra, não há necessidade de sincronização ou \textit{mutex}. Tornar o programa paralelizável é simples já que requer um uso mínimo do ferramental de programação concorrente.


  Toda a vez que um fóton é criado logo em seguida sua rota é traçada, a relação entre criação e cálculo de trajetória é de 1 para 1. O que é evidente ao se observar as linhas 106 a 137 do \textit{grmonty.cu}, aqui copiadas:

  \label{sec:main_loop}
  \begin{lstlisting}
    #pragma omp parallel private(ph)
  	{
  		while (1) {
  			/* get pseudo-quanta */
    #pragma omp critical (MAKE_SPHOT)
  		  {
  				if (!quit_flag)
  					make_super_photon(&ph, &quit_flag);
  			}
  			if (quit_flag)
  				break;

  			/* push them around */
  			track_super_photon(&ph);

  			/* step */
    #pragma omp atomic
  			N_superph_made += 1;
        /*mais codigo*/
  		}
  	}
  \end{lstlisting}

  Fica claro também - ao observar a linha 8 a 14 - que o processamento do rastreio é feito assim que possível, ao oposto de um processamento em lotes, ou seja, assim que o comandado \textit{make\_super\_photon} é executado, gerando um novo fóton \textit{ph}, o  \textit{track\_super\_photon} é chamado, não havendo algum buffer ou lote, um fóton produzido é um fóton consumido.

  Tal processamento reduz muito os vestígios que um fóton pode criar durante sua existência. Uma vez que não se perde tempo deixando-o na memória, assim que é mensurado seu espaço na memória já é ocupado pelo próximo fóton a ser produzido, há um foco na economia de memória. O número de fótons na memória é o número de threads rodando simultaneamente.

  Por fim se faz necessário notar que o programa é escrito na linguagem de programação C. O que faz muito sentido do ponto de vista de performance, uma vez que C é uma linguagem de baixo nível, mais próxima a linguagem de máquina e por isso é quase sempre explícito a quantidade e de que forma se está manipulando a memória. Outras vantagens são as possibilidades de usar tanto a biblioteca \textbf{OpenMP} como as otimizações do \textbf{gcc}, o \textit{Gnu C Compiler}, mas do ponto de vista da expressividade uma linguagem de mais alto nível poderia apresentar outras vantagens, como uma maior legibilidade do código e o uso de abstrações e encapsulamento, aumentando também a capacidade e a facilidade de fazer manutenções e melhorias no código.
