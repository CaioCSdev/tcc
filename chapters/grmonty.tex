\chapter{Grmonty: Monte Carlo para Relatividade Geral}
\label{cap:grmonty}

\section{O que Faz}
  Dolence et al definem o \textbf{grmonty} como ``software destinado a calcular o espectro de plasmas quentes e opticamente finos a par da completa relatividade geral utilizando um código de transporte radioativo baseado na técnica de Monte Carlo''\citep[p.1, traduzido]{Dolence:09}. Em outras palavras o programa estima o espectro de uma simulação de magnetoidrodinâmica relativistica utilizando o método de Monte Carlo para aproximar o espectro.

  Utilizando o método de Monte Carlo na geração dos dados, os fótons, e a partir de um dado modelo de plasma fornecido como entrada, o programa busca gerar o espectro de radiação. Para tanto um número próximo a \textit{N} - fornecido na entrada - de fótons é gerado e para cada fóton sua tragetória é traçada. Nessa trajetória o fóton é espalhado e passa por diferentes interações, podendo ser absorvido, refletido, difratado ou até re-emitido ao percorrer seu percurso e finalmente ser medido.

  Depois de algumas iterações um número próximo a \textit{N} de fótons já foi gerado e rastreado, assim um relatório com o espectograma obtido é gerado e retornado pelo programa que finalmente termina.

\section{Para que Faz}
  Qualquer fonte astrofísica de radiação que seja relativistica, ou seja, qualquer corpo ou fenômeno fonte de radiação eletromagnética, seja do rádio à raios gama e que é relativistica: apresenta uma considerável distorção no espaço-tempo, seja por estar em velocidades próximas a da luz, seja por possuir uma enorme quantidade de massa e/ou energia. Exemplos de objetos são os buracos negros e estrelas de neutrons, fenômenos são os \textit{Gamma Ray Bursts} ou núcleos ativos de galáxias.

  Foram desenvolvidas várias técnicas para calcular a transferência radioativa a partir de fontes como as descritas a cima\citep{Dolence:09}, porém poucas levam em conta a relatividade geral como um todo, pricipalmente no quesíto de objetos, fontes, muito massivas ou com velocidades próxima a da luz, o \textbf{grmonty} vem para aprimorar esses cálculos.

\section{Como Faz}

  No momento de criação e rastreio dos fótons o programa faz o uso da biblioteca \textbf{OpenMP} para paralelizar o desenvolvimento dos fótons, graças a esta abordagem é viável o potêncial uso de todos os núcleos disponíveis na CPU da máquina. A bilioteca é utilizada para que cada fóton seja produzido e espalhado de forma independente dos outros e funcionando em paralelo, além disso todas as intruções não dependem do fóton em sí, elas são as mesmas intruções para todos os fótons. Desta forma podemos caracterizar o \textbf{grmonty} como tendo uma computação SIMD.

  ``Única Instrução Multiplos Dados: Nesse tipo de computação podem haver multiplos processadors, cada um operando sobre seu item de dados, mas estão todos executando a mesma instrução naquele item de dados''\citep[p.84, traduzido]{HCP:16}


  Assim que um fóton é gerado seu percurso é traçado. O que é evidente ao se observar as linhas 106 a 137 do \textit{grmonty.cu}, aqui copiadas:

  \begin{lstlisting}
    #pragma omp parallel private(ph)
  	{
  		while (1) {
  			/* get pseudo-quanta */
    #pragma omp critical (MAKE_SPHOT)
  		  {
  				if (!quit_flag)
  					make_super_photon(&ph, &quit_flag);
  			}
  			if (quit_flag)
  				break;

  			/* push them around */
  			track_super_photon(&ph);

  			/* step */
    #pragma omp atomic
  			N_superph_made += 1;
        /*mais codigo*/
  		}
  	}
  \end{lstlisting}

  Fica claro também - ao observar a linha 8 a 14 - que o processamento do rastreio é feito assim que possível, ao oposto de um processamento em lotes, ou seja, assim que o comandado \textit{make\_super\_photon} é executado gerando um novo fóton \textit{ph}, o  \textit{track\_super\_photon} é chamado, não havendo algum buffer ou lote, um fóton produzido é um fóton consumido.

  % descrever funcionamento computacional, MPI, SIMD, citar assim que possível oposto de lazy ou batch, C
