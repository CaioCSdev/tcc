\chapter{Conclusões}
\label{cap:conclusoes}

A petrformance do grmonty está maior, ele responde mais rápido do que antiormente, a utilização de GPGPU trouxe uma maior complexidade ao progranma mas encontra partida sua velocidade é muitomaiior. Ainda existe muito a aser feito para melhor ar a performance, seja redimensionando o tamanho dos batches a serem processados a uam serialização mais rigorosa na n os kernel, pois como ainda exitem recusrsoes -mexistem caso que a track suoper fonton chama-se a si mesmo para recalcular uma a trajetoria - e esses comportamentios afentam muiitro a perfoamnace pois todas as threads ficam esperando enquanto uma deles vai executar o tracks usper foton todo novamente.

De a cordo com o caras do massively um ganho de 10 vezes ṕe esperado em uma primeira trandformação do cṕodigo para a gpu , mas ganho de até 100x pode ser atingidos se um processo mais fino e delicado for aplicado.



Calculos astrofisicos são importantes e o tempo que tomam para retornarem um resultado pode ser um limitante para a veloccidade coma qual a ciência progride.

Sistemas gigatesco e caros são construido como  unico proposito de computatar dados para que se possam realizar experimentos, então é um enorme passo quando um pesquisador pode fazer tais calculos muito complexos compuitacionamente no mesmo computador onde joga seus jogos digitais.

o avanço da ciência depende de artiteturas de alta performance, gpus tem se apresentado competentes na realização de tais tarefas, e sua popular adoção facilita um maior acesso computação astrofísica, aumentando assim a velocidade do progresso cientifico.
