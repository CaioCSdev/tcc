\chapter{Conclusões}
\label{cap:conclusoes}

Buscou-se nesse trabalho aplicar conhecimentos de arquiteturas de processadores, mais especificamente das GPGPUs, para aumentar a performance e diminuir o tempo de espera que o programa grmonty precisa para trazer resultados satisfatórios. Durante o seu desenvolvimento experimentos foram realizados e dados foram coletados resultando em um aparente \textit{speed up} de 3 vezes, contribuindo para a ideia que a portabilidade de códigos para GPGPUS é uma forma válida e satisfatória para um aumento de performance para programas que podem ser altamente paralelizáveis.

A performance do grmonty está maior, ele responde mais rápido do que anteriormente. A utilização da GPGPU trouxe uma maior complexidade ao programa mas e contrapartida sua velocidade é muito maior. Ainda existe muito que pode ser feito para melhorar a performance, seja redimensionando o tamanho dos batches a serem processados seja uma serializadão mais rigorosa nos kernels. Ainda exitem recursões como no caso do \textit{track\_super\_photon} que chama a si mesmo para recalcular uma a trajetória caso passe por uma iteração que o espalhe novamente. Esses comportamentos afetam muito a performance pois todas as threads ficam esperando enquanto uma deles vai executar o \textit{track\_super\_photon} todo novamente.

De a cordo com \citep{massively:16} um ganho de 10 vezes é esperado em uma primeira transformação do código para a gpu, mas ganhos de até 100 vezes podem ser atingidos se um processo mais fino e delicado for aplicado, minimizando a alocação de memória na GPGPU e tornando a execução com a menor quantidade de desvios de fluxo possível, buscando tornar as threads mais uniformes que possível.

Cálculos astrofísicos são importantes e o tempo que tomam para retornarem um resultado pode ser um limitante para a velocidade coma qual a ciência progride.

Sistemas gigantesco e de alto custo são construídos com o único proposito de computar dados para que se possam realizar experimentos, então é um enorme passo quando um pesquisador pode fazer tais cálculos muito complexos computacionalmente no mesmo computador onde joga seus jogos digitais no fim de semana.

O avanço da ciência depende de arquiteturas de alta performance, GPGPUs tem se apresentado competentes na realização de tais tarefas, e sua popular adoção facilita um maior acesso computação astrofísica, aumentando assim a velocidade do progresso cientifico.
