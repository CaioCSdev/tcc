%% ------------------------------------------------------------------------- %%
\chapter{Introdução}
\label{cap:introducao}

apresentar o que e o grmonty, motivações para aumentar a sua performance.

mas como aumentar a velocidade? dada a arquitetura de funcionamento do programa uma ótima estrategia eh mover parte de seu funcionamento para um GPU, pois o programa pode sr visto como uma SIMD

Uma grande dúvida dos astrofísicos e também de toda a comunidade científica é o que ocorre em um buraco negro e em suas proximidades. Na busca de respostas programas de computador são feitos com o intuito de simular essa região e talvez trazer alguma luz, um desses programas é o \emph{GrMonty}.

Programas dessa natureza tendem a ser muito intensos no que diz respeito ao processamento, exigindo muito das CPUs estas tornam-se assim um limitante, um gargalo, para a velocidade com a qual o programa pode devolver um resultado. É neste contexto que buscamos aplicar métodos de \emph{Computação de Alta Performance} para otimizar ao máximo o uso todos os dispositivos do computador (hardware) que temos disponíveis.

Muitas das técnicas de HPC exploram a paralelização, o que pode ser feito massivamente por um hardware específico as \emph{unidades de processamento gráfico}, GPU. Tais dispositivos são confeccionados primordialmente para processamento gráfico em jogos digitais, porém graças a avanços recentes tais dispositivos tem se tornado mais genéricos

Ao analisar o funcionamento do \emph{GrMonty}, por sua característica de simulador de partículas, é possível classificar parte de sua execução no modelo SIMD. Dada essa informação podemos explorar


Aumento de performance em cálculos de transferência radioativa ao redor de
buracos negros usando unidades de processamento gráfico de propósito geral
(GPGPU).
GRmonty (nome reduzido, em inglês, de Monte Carlo para Relatividade Geral) é um programa de computador que simula as características ao redor de um buraco negro afim de calcular a transferência radioativa,


Existe o programa grmonty \citep{Dolence:09},
o qual calcula a propagação de fótons nas proximidades de um buraco negro,
utilizando-se de um alto custo de processamento. Apesar desse programa estar
escrito em C e utilizando bibliotecas de computação em alta performance
(HPC sigla em inglês) como OpenMP ainda há muito espaço para melhorias em sua
performance. Este trabalho mira ser uma dessas melhorias.

A estrutura de funcionamento do grmonty por ser descrita como um único cálculo
aplicado a uma grande quantidade de dados, no qual cada calculo é independente
dos demais. Dada esta descrição é possível usar um modelo de computação, única
instrução múltiplos dados (SIMD), com um hardware especializado a fim de
aumentar a performance do calculo como um todo. O emprego desta técnica neste
programa específico é o tema deste TCC.

Uma monografia deve ter um capítulo inicial que é a Introdução e um
capítulo final que é a Conclusão. Entre esses dois capítulos poderá
ter uma sequência de capítulos que descrevem o trabalho em detalhes.
Após o capítulo de conclusão, poderá ter apêndices e ao final deverá
ter as referências bibliográficas.
