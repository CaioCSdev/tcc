\chapter{Introdução}
\label{cap:introducao}

Uma grande dúvida dos astrofísicos e também de toda a comunidade científica é o que ocorre em um buraco negro e em suas proximidades. Na busca de respostas, programas de computador são feitos com o intuito de simular essa região e talvez trazer alguma luz. Um desses programas é o \textbf{grmonty}\citep{Dolence:09} (nome reduzido, em inglês, de Monte Carlo para Relatividade Geral).

Programas dessa natureza tendem a ser muito intensos no que diz respeito ao processamento, exigindo muito das CPUs. Estas tornam-se assim um limitante, um gargalo, para a velocidade com a qual o programa pode devolver um resultado. É neste contexto que buscamos aplicar métodos de \emph{Computação de Alta Performance} para otimizar ao máximo o uso de todos os dispositivos do computador (hardware) que temos disponíveis.

Muitas das técnicas de HPC exploram a paralelização, o que pode ser feito massivamente por um hardware específico, nesse caso as \emph{unidades de processamento gráfico}, GPU. Tais dispositivos são muito populares e já presentes em muitas máquinas domésticas e até em smartphones. Eles são confeccionados primordialmente para processamento gráfico em jogos digitais, porém, graças aos avanços recentes tais dispositivos vêm se tornando mais genéricos e respondendo a uma gama maior de problemas.

Ao analisar o funcionamento do \textbf{grmonty} - por sua característica de simulador de partículas - é possível classificar parte de sua execução no modelo SIMD (sigla em inglês para única instrução múltiplos dados), uma vez que simula a trajetória da cada fóton de maneira independente. Dada essa informação, podemos explorar o poder computacional das GPUs com o intuito de paralelizar a execução do código, aumentando drasticamente sua a performance.

Existem vários programas que simulam essas regiões próximas a buracos negros, porém, poucos tem a amplitude e relacionam diferentes propriedades físicas, em especial a relatividade geral, como o \textbf{grmonty}. Tornar o programa mais performático e lhe fornecer a capacidade de executar em computadores domésticos com dados e precisão relevantes, prestaria um avanço na pesquisa de buracos negros. Facilitando assim a execução de simulações e diminuindo o tempo de espera por resultados.

O \textbf{grmonty} também apresenta algumas outras características que o tornam atraente no quesito de programa que pode ser otimizado. Como ele é relativamente pequeno (menos de 5 mil linhas), possuir a característica de ter uma arquitetura SIMD e ser feito todo em linguagem C, uma portabilidade para ser executado em GPUs é um ato factível e que pode apresentar grandes ganhos com um esforço não muito alto, aproveitando assim da paralelização massiva que as placas gráficas apresentam.

Este trabalho tem como objetivo apresentar melhorias a execução do código do \textbf{grmonty}, utilizando-se do processador de placas gráficas, as GPUs, para massivamente paralelizar e distribuir a carga de trabalho pelos múltiplos núcleos de processamento destes dispositivos. Primeiramente é explicado o que é o \textbf{grmonty} e como funciona, depois os paradigmas ao qual sua execução apoia-se, caminhando para a explicação de GPUs e como contribuem para o aumento do desempenho, assim são apresentadas as otimizações executadas, chegando finalmente nos resultados alcançados e conclusões.
