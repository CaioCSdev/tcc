\chapter{Introdução}
\label{cap:introducao}

Uma grande dúvida dos astrofísicos e também de toda a comunidade científica é o que ocorre em um buraco negro e em suas proximidades. Na busca de respostas programas de computador são feitos com o intuito de simular essa região e talvez trazer alguma luz, um desses programas é o \textbf{grmonty}\citep{Dolence:09} (nome reduzido, em inglês, de Monte Carlo para Relatividade Geral).

Programas dessa natureza tendem a ser muito intensos no que diz respeito ao processamento, exigindo muito das CPUs, estas tornam-se assim um limitante, um gargalo, para a velocidade com a qual o programa pode devolver um resultado. É neste contexto que buscamos aplicar métodos de \emph{Computação de Alta Performance} para otimizar ao máximo o uso todos os dispositivos do computador (hardware) que temos disponíveis.

Muitas das técnicas de HPC exploram a paralelização, o que pode ser feito massivamente por um hardware específico nesse caso as \emph{unidades de processamento gráfico}, GPU. Tais dispositivos são muito populares e já presentes em muitas máquinas domésticas e até em smartphones, eles são confeccionados primordialmente para processamento gráfico em jogos digitais, porém graças aos avanços recentes tais dispositivos tem se tornado mais genéricos e respondendo a uma gama maior de problemas.

Ao analisar o funcionamento do \textbf{grmonty} - por sua característica de simulador de partículas - é possível classificar parte de sua execução no modelo SIMD (sigla em inglês para única instrução múltiplos dados), uma vez que simula a trajetória da cada fóton de maneira independente. Dada essa informação podemos explorar o poder computacional das GPUs afim de paralelizar a execução do código, aumentando drasticamente sua a performance.

Existem vários programas que simulam essas regiões próximas a buracos negros, porém poucos tem a amplitude e relacionam diferentes propriedades físicas, em especial a relatividade geral, como o \textbf{grmonty}. Tornando o programa mais performático e lhe fornecendo a capacidade de executar em computadores domésticos com dados e precisão relevantes, prestaria um grande avanço na pesquisa de buracos negros, facilitando a execução de simulações e dimanado o tempo de espera por dados.

O \textbf{grmonty} também apresenta algumas outras características que o tornam mais atraente no quesito de programa que pode ser otimizado. Como ele é relativamente pequeno ( menos de 5 mil linhas ), possuir a característica de ter uma arquitetura SIMD e ser feito todo em linguagem C, uma portabilidade para ser executado em GPUs é um ato factível e que pode apresentar grandes ganhos com um esforço não muito alto, aproveitando assim da paralelização massiva que as placas gráficas apresentam.

Este trabalho tem como objetivo apresentar melhorias a execução do código do \textbf{grmonty}, utilizando-se do processador de placas gráficas, as GPUs, para massivamente paralelizar e distribuir a carga de trabalho pelos múltiplos núcleos de processamento destas placas. Primeiramente é explicado o que é o \textbf{grmonty} e como funciona, depois os paradigmas ao qual sua execução apoia-se, caminhando para a explicação de GPUs e como contribuem para o aumento de performance, assim são apresentadas as otimizações executadas, chegando finalmente nos resultados alcançados e conclusões. Há ainda um capítulo de próximos passos demonstrando que ainda há muito espaço para mais melhorias e mais velocidade na execução.

%
% apresentar o que e o grmonty, motivações para aumentar a sua performance.
%
% mas como aumentar a velocidade? dada a arquitetura de funcionamento do programa uma ótima estrategia eh mover parte de seu funcionamento para um GPU, pois o programa pode sr visto como uma SIMD
%
%
% Aumento de performance em cálculos de transferência radioativa ao redor de
% buracos negros usando unidades de processamento gráfico de propósito geral
% (GPGPU).
% GRmonty (nome reduzido, em inglês, de Monte Carlo para Relatividade Geral) é um programa de computador que simula as características ao redor de um buraco negro afim de calcular a transferência radioativa,
%
%
% Existe o programa grmonty \citep{Dolence:09},
% o qual calcula a propagação de fótons nas proximidades de um buraco negro,
% utilizando-se de um alto custo de processamento. Apesar desse programa estar
% escrito em C e utilizando bibliotecas de computação em alta performance
% (HPC sigla em inglês) como OpenMP ainda há muito espaço para melhorias em sua
% performance. Este trabalho mira ser uma dessas melhorias.
%
% A estrutura de funcionamento do grmonty por ser descrita como um único cálculo
% aplicado a uma grande quantidade de dados, no qual cada calculo é independente
% dos demais. Dada esta descrição é possível usar um modelo de computação, única
% instrução múltiplos dados (SIMD), com um hardware especializado a fim de
% aumentar a performance do calculo como um todo. O emprego desta técnica neste
% programa específico é o tema deste TCC.
%
% Uma monografia deve ter um capítulo inicial que é a Introdução e um
% capítulo final que é a Conclusão. Entre esses dois capítulos poderá
% ter uma sequência de capítulos que descrevem o trabalho em detalhes.
% Após o capítulo de conclusão, poderá ter apêndices e ao final deverá
% ter as referências bibliográficas.
