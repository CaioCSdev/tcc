% Arquivo LaTeX de exemplo de monografia para a disciplina MAC0499
%
% Adaptado em julho/2015 a partir do
%
% ---------------------------------------------------------------------------- %
% Arquivo LaTeX de exemplo de dissertação/tese a ser apresentados à CPG do IME-USP
%
% Versão 5: Sex Mar  9 18:05:40 BRT 2012
%
% Criação: Jesús P. Mena-Chalco
% Revisão: Fabio Kon e Paulo Feofiloff


\documentclass[12pt,twoside,a4paper]{book}


% ---------------------------------------------------------------------------- %
% Pacotes
\usepackage[utf8]{inputenc}
\usepackage[T1]{fontenc}
\usepackage[brazil]{babel}
\usepackage[pdftex]{graphicx}           % usamos arquivos pdf/png como figuras
\usepackage{setspace}                   % espaçamento flexível
\usepackage{indentfirst}                % indentação do primeiro parágrafo
\usepackage{makeidx}                    % índice remissivo
\usepackage[nottoc]{tocbibind}          % acrescentamos a bibliografia/indice/conteudo no Table of Contents
\usepackage{courier}                    % usa o Adobe Courier no lugar de Computer Modern Typewriter
\usepackage{type1cm}                    % fontes realmente escaláveis
\usepackage{listings}                   % para formatar código-fonte (ex. em Java)
\usepackage{titletoc}
%\usepackage[bf,small,compact]{titlesec} % cabeçalhos dos títulos: menores e compactos
\usepackage[fixlanguage]{babelbib}
\usepackage[font=small,format=plain,labelfont=bf,up,textfont=it,up]{caption}
\usepackage[usenames,svgnames,dvipsnames]{xcolor}
\usepackage[a4paper,top=2.54cm,bottom=2.0cm,left=2.0cm,right=2.54cm]{geometry} % margens
%\usepackage[pdftex,plainpages=false,pdfpagelabels,pagebackref,colorlinks=true,citecolor=black,linkcolor=black,urlcolor=black,filecolor=black,bookmarksopen=true]{hyperref} % links em preto
\usepackage[pdftex,plainpages=false,pdfpagelabels,pagebackref,colorlinks=true,citecolor=DarkGreen,linkcolor=NavyBlue,urlcolor=DarkRed,filecolor=green,bookmarksopen=true]{hyperref} % links coloridos
\usepackage[all]{hypcap}                    % soluciona o problema com o hyperref e capitulos
\usepackage[round,sort,nonamebreak]{natbib} % citação bibliográfica textual(plainnat-ime.bst)
\usepackage{emptypage}  % para não colocar número de página em página vazia
\usepackage{rotating}
\fontsize{60}{62}\usefont{OT1}{cmr}{m}{n}{\selectfont}

% ---------------------------------------------------------------------------- %
% Cabeçalhos similares ao TAOCP de Donald E. Knuth
\usepackage{fancyhdr}
\pagestyle{fancy}
\fancyhf{}
\renewcommand{\chaptermark}[1]{\markboth{\MakeUppercase{#1}}{}}
\renewcommand{\sectionmark}[1]{\markright{\MakeUppercase{#1}}{}}
\renewcommand{\headrulewidth}{0pt}

% ---------------------------------------------------------------------------- %
\graphicspath{{./figuras/}}             % caminho das figuras (recomendável)
\frenchspacing                          % arruma o espaço: id est (i.e.) e exempli gratia (e.g.)
\urlstyle{same}                         % URL com o mesmo estilo do texto e não mono-spaced
\makeindex                              % para o índice remissivo
\raggedbottom                           % para não permitir espaços extra no texto
\fontsize{60}{62}\usefont{OT1}{cmr}{m}{n}{\selectfont}
\cleardoublepage
\normalsize

% ---------------------------------------------------------------------------- %
% Opções de listing usados para o código fonte
% Ref: http://en.wikibooks.org/wiki/LaTeX/Packages/Listings
\lstset{ %
language=C,                     % choose the language of the code
basicstyle={\small\ttfamily},       % the size of the fonts that are used for the code
numbers=left,                   % where to put the line-numbers
numberstyle=\footnotesize,      % the size of the fonts that are used for the line-numbers
stepnumber=1,                   % the step between two line-numbers. If it's 1 each line will be numbered
numbersep=5pt,                  % how far the line-numbers are from the code
showspaces=false,               % show spaces adding particular underscores
showstringspaces=false,         % underline spaces within strings
showtabs=false,                 % show tabs within strings adding particular underscores
frame=single,	                % adds a frame around the code
framerule=0.6pt,
tabsize=2,	                    % sets default tabsize to 2 spaces
captionpos=b,                   % sets the caption-position to bottom
breaklines=true,                % sets automatic line breaking
breakatwhitespace=false,        % sets if automatic breaks should only happen at whitespace
escapeinside={\%*}{*)},         % if you want to add a comment within your code
backgroundcolor=\color[rgb]{1.0,1.0,1.0}, % choose the background color.
rulecolor=\color[rgb]{0.8,0.8,0.8},
extendedchars=true,
xleftmargin=10pt,
xrightmargin=10pt,
framexleftmargin=10pt,
framexrightmargin=10pt,
commentstyle=\color[rgb]{0,0.6,0},
keywordstyle=\color{blue}
}

% ---------------------------------------------------------------------------- %
% Corpo do texto
\begin{document}

\frontmatter
% cabeçalho para as páginas das seções anteriores ao capítulo 1 (frontmatter)
\fancyhead[RO]{{\footnotesize\rightmark}\hspace{2em}\thepage}
\setcounter{tocdepth}{2}
\fancyhead[LE]{\thepage\hspace{2em}\footnotesize{\leftmark}}
\fancyhead[RE,LO]{}
\fancyhead[RO]{{\footnotesize\rightmark}\hspace{2em}\thepage}

\onehalfspacing  % espaçamento

% ---------------------------------------------------------------------------- %
% CAPA
\thispagestyle{empty}
\begin{center}
    \vspace*{2.3cm}
    Universidade de São Paulo\\
    Instituto de Matemática e Estatística\\
    Bacharelado em Ciência da Computação


    \vspace*{3cm}
    \Large{Caio Costa Salgado}


    \vspace{3cm}
    \textbf{\Large{Acelerando transporte radiativo ao redor de buracos negros com GPUs}}


    \vskip 5cm
    \normalsize{São Paulo}

    \normalsize{Dezembro de 2017}
\end{center}

% ---------------------------------------------------------------------------- %
% Página de rosto
%
\newpage
\thispagestyle{empty}
  \begin{center}
    \vspace*{2.3 cm}
    \textbf{\Large{Acelerando transporte radiativo ao redor de buracos negros com GPUs}}
    \vspace*{2 cm}
  \end{center}

  \vskip 2cm

  \begin{flushright}
    Monografia final da disciplina \\
    MAC0499 -- Trabalho de Formatura Supervisionado.
  \end{flushright}

  \vskip 5cm

  \begin{center}
    Supervisor: Prof. Dr. Rodrigo Nemmen da Silva

    \vskip 5cm
    \normalsize{São Paulo}

    \normalsize{Dezembro de 2017}
  \end{center}
\pagebreak

\pagenumbering{roman}     % começamos a numerar

% ---------------------------------------------------------------------------- %
% Resumo

\chapter*{Resumo}

  Grmonty é um programa para o cálculo do espectro radioativo próximo a buracos negros, mas afim de executá-lo é necessário paciência pois sua estrutura é muito onerosa á CPU da máquina. Com o intuito de melhorar a performance do programa uma técnica de Computação de Alta Performance é empregada, a utilização de processadores massivamente paralelos, no caso as GPGPUs para distribuir a carga de trabalho e aumentar a velocidade de execução. Com esta técnica demonstramos uma aparente melhora em 3 vezes na velocidade de execução, fornecendo indícios que a técnica pode trazer grandes vantagens a performance de um programa.
\\

\noindent \textbf{Palavras-chave:} GPGPU, CUDA, HPC, Monte Carlo, Transferência radioativa, Buraco Negro.

\chapter*{Abstract}
  Grmonty is a software made for the calculus o radiactive spectrum near black holes, but in order to execute it pacience is needed since it's struct takes to hash with the CPU of the machine. With the ideia in mind of improving the performance of the software a tecnic from High Performance computing is applied, the use of massively parallel processors, in this case the GPGPUS to distribute the worload and increase the speed of execution. with this tecnic we show what seems to be a 3 times improve im speed of execution, providing evidence that the tecnic may bring gains in performance for one's program.
\\

\noindent \textbf{Keywords:} GPGPU, CUDA, HPC, Monte Carlo, Rasioactive Transfer, Black Hole.


\tableofcontents

%%-------------------------------------------------------------------------- %
\chapter{Lista de Abreviaturas}
\begin{tabular}{ll}
  GPU   & Unidade de processamento Gráfico (\emph{Graphics Grocessing Unit})\\

  CPU   & Unidade de Processamento Central (\emph{Central Processing Unit})\\

  GPGPU & Unidade de processamento Gráfico de Propósito Geral\\
        & (\emph{General Purpose Graphics Processing Unit})\\

  CUDA  & Computação em Arquitetura unificada de dispositivos\\
        & (\emph{Compute Unified Device Architecture})\\

  HCP   & Computação de Alta Performance (\emph{High Performance Computing})\\

  SIMD  & Única Instrução Múltiplos dados (\emph{Sigle Instruction Multiple Dada})\\
\end{tabular}

%% % ---------------------------------------------------------------------------- %
%% \chapter{Lista de Símbolos}
%% \begin{tabular}{ll}
%%         $\omega$    & Frequência angular\\
%%         $\psi$      & Função de análise \emph{wavelet}\\
%%         $\Psi$      & Transformada de Fourier de $\psi$\\
%% \end{tabular}

%% % ---------------------------------------------------------------------------- %
%% % Listas de figuras e tabelas criadas automaticamente
%% \listoffigures
%% \listoftables



% ---------------------------------------------------------------------------- %
% Capítulos do trabalho
\mainmatter

% cabeçalho para as páginas de todos os capítulos
\fancyhead[RE,LO]{\thesection}

% \singlespacing              % espaçamento simples
\onehalfspacing            % espaçamento um e meio

\input chapters/introducao
\input chapters/grmonty
\input chapters/gpgpu
\input chapters/otimizacao
\input chapters/resultados
\input chapters/conclusoes

% cabeçalho para os apêndices
\renewcommand{\chaptermark}[1]{\markboth{\MakeUppercase{\appendixname\ \thechapter}} {\MakeUppercase{#1}} }
\fancyhead[RE,LO]{}
\appendix

\chapter{Título do apêndice}
\label{cap:ape}

Texto texto texto texto texto texto texto texto texto texto texto texto texto
texto texto texto texto texto texto texto texto texto texto texto texto texto
texto texto texto texto texto texto.

      % associado ao arquivo: 'ape-conjuntos.tex'


% ---------------------------------------------------------------------------- %
% Bibliografia
\backmatter \singlespacing   % espaçamento simples
\bibliographystyle{plainnat-ime} % citação bibliográfica textual
\bibliography{bibliografia}  % associado ao arquivo: 'bibliografia.bib'


%%%  ---------------------------------------------------------------------------- %
%% % Índice remissivo
%% \index{TBP|see{periodicidade região codificante}}
%% \index{DSP|see{processamento digital de sinais}}
%% \index{STFT|see{transformada de Fourier de tempo reduzido}}
%% \index{DFT|see{transformada discreta de Fourier}}
%% \index{Fourier!transformada|see{transformada de Fourier}}

%% \printindex   % imprime o índice remissivo no documento

\end{document}
