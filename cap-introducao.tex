%% ------------------------------------------------------------------------- %%
\chapter{Introdução}
\label{cap:introducao}

Aumento de performance em cálculos de transferência radioativa ao redor de
buracos negros usando unidades de processamento gráfico de propósito geral
(GPGPU).
GRmonty (nome reduzido, em inglês, de Monte Carlo para Relatividade Geral) é um programa de computador que simula as caractéristicas ao redor de um buraco negro afim de calcular a transferência radioativa,


Existe o programa grmonty ( https://github.com/rsnemmen/grmonty Dolence et al. 2009 ApJ),
o qual calcula a propagação de fótons nas proximidades de um buraco negro,
utilizando-se de um alto custo de processamento. Apesar desse programa estar
escrito em C e utilizando bibliotecas de computação em alta performance
(HPC sigla em inglês) como OpenMP ainda há muito espaço para melhorias em sua
performance. Este trabalho mira ser uma dessas melhorias.

A estrutura de funcionamento do grmonty por ser descrita como um único cálculo
aplicado a uma grande quantidade de dados, no qual cada calculo é independente
dos demais. Dada esta descrição é possível usar um modelo de computação, única
instrução múltiplos dados (SIMD), com um hardware especializado a fim de
aumentar a performance do calculo como um todo. O emprego desta técnica neste
programa específico é o tema deste TCC.

Uma monografia deve ter um capítulo inicial que é a Introdução e um
capítulo final que é a Conclusão. Entre esses dois capítulos poderá
ter uma sequência de capítulos que descrevem o trabalho em detalhes.
Após o capítulo de conclusão, poderá ter apêndices e ao final deverá
ter as referências bibliográficas.


Para a escrita de textos em Ciência da Computação, o livro de Justin Zobel,
\emph{Writing for Computer Science} \citep{zobel:04} é uma leitura obrigatória.
O livro \emph{Metodologia de Pesquisa para Ciência da Computação} de
\citet{waz:09} também merece uma boa lida.

O uso desnecessário de termos em língua estrangeira deve ser evitado. No entanto,
quando isso for necessário, os termos devem aparecer \emph{em itálico}.

\begin{small}
\begin{verbatim}
Modos de citação:
indesejável: [AF83] introduziu o algoritmo ótimo.
indesejável: (Andrew e Foster, 1983) introduziram o algoritmo ótimo.
certo : Andrew e Foster introduziram o algoritmo ótimo [AF83].
certo : Andrew e Foster introduziram o algoritmo ótimo (Andrew e Foster, 1983).
certo : Andrew e Foster (1983) introduziram o algoritmo ótimo.
\end{verbatim}
\end{small}

Uma prática recomendável na escrita de textos é descrever as legendas das
figuras e tabelas em forma auto-contida: as legendas devem ser razoavelmente
completas, de modo que o leitor possa entender a figura sem ler o texto onde a
figura ou tabela é citada.

Apresentar os resultados de forma simples, clara e completa é uma tarefa que
requer inspiração. Nesse sentido, o livro de \citet{tufte01:visualDisplay},
\emph{The Visual Display of Quantitative Information}, serve de ajuda na
criação de figuras que permitam entender e interpretar dados/resultados de forma
eficiente.
