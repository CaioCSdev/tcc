\chapter{Resumo1}
\label{cap:resumo2}

Uma grande dúvida dos astrofísicos e também de toda a comunidade ciêntifica é o que ocorre em um buraco negro e em suas proximidades. Na busca de respostas programas de computador são feitos com o intuito de simular essa região e talvez trazer alguma luz, um desses programas é o \emph{GrMonty}.

Programas dessa natureza tendem a ser muito intensos no que diz respeito ao processamento, exigindo muito das \emph{unidades centrais de processamento} (CPU, em inglês) estas tornam-se assim um limitante, um gargalo, para a velocidade com a qual o programa pode devolver um resultado. É neste contexto que buscamos aplicar métodos de \emph{Computação de Alta Performance} (HPC, em inglês) para otimizar ao máximo o uso todos os dispositivos do computador (hardware) que temos disponíveis.

Muitas das técnicas de HPC exploram a paralelização, o que pode ser feito massivamente por um hardware expecífico as \emph{unidades de processamento gráfico} (GPU, em inglês). Tais dispositivos são confeccionados primordialmente para processamento gráfico em jogos digitais, porém graças a avanços recentes tais dispositivos tem se tornado mais genéricos

Ao analisar o funcionamento do \emph{GrMonty}, por sua característica de simulador de partículas, é possível classificar parte de sua execução no modelo "única instrução, múltiplos dados" (SIMD, em inglês). Dada essa informação podemos explorar
